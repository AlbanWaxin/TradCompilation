\documentclass[12pt,twocolumn]{report}
\usepackage[utf8]{inputenc}
\usepackage[T1]{fontenc}
\usepackage[fleqn]{amsmath}
\usepackage{amsfonts,amssymb,stmaryrd}
\usepackage[english]{babel}
\usepackage{pdfpages}

%=============Affichage=======================
\usepackage{fullpage}
\usepackage{mathtools}
\usepackage{lmodern}
\usepackage{xcolor}
\usepackage{enumitem}
\usepackage{tikz,tkz-tab}
\usepackage[ruled,vlined]{algorithm2e}
\usepackage{booktabs}
\usepackage{setspace}
\usepackage{tikz}
\usetikzlibrary{arrows,automata}
\title{Règles de traduction Compilation}
\date{}

\setlength{\topmargin}{-1.5cm}
\setlength{\textheight}{25cm}
\setlength{\textwidth}{20.5cm}
\setlength{\oddsidemargin}{-1.5cm}
\setlength{\evensidemargin}{50cm}
\doublespacing  

\newcommand{\rd}[1]{\textcolor{red}{#1}}
\newcommand{\g}[1]{\textcolor{lime}{#1}}
\newcommand{\dg}[1]{\textcolor{dgreen}{#1}}
\newcommand{\blue}[1]{\textcolor{blue}{#1}}
\newcommand{\cy}[1]{\textcolor{cyan}{#1}}
\newcommand{\blz}{$\blacklozenge$}
\newcommand{\ns}{\\\indent\indent\vspace{0.25cm}}
\setcounter{secnumdepth}{5}% profondeur de la table des matières
\usepackage{titlesec}


\titleformat{\chapter}[frame]
{\Huge}
{\filright\rmfamily\bfseries\Huge\enspace\thechapter\enspace}
{18pt}
{\rmfamily\huge\bfseries\filcenter}
% rmfamily=roman, sffamily = sans serif ou ttfamily =type writer
\usepackage[many]{tcolorbox} % Creation de box collorable pour le texte non intégré

\newtcolorbox{trad}{colback=white,colframe=black,arc=0mm,sharp corners= northwest,arc=10pt}

%=============Mathématiques=================

%--------------Raccourcis:------------------

\renewcommand{\o}{\circ}
\newcommand{\abs}[1]{\left\lvert#1\right\rvert}
\DeclarePairedDelimiter{\ceil}{\lceil}{\rceil}
\renewcommand*{\overrightarrow}[1]{\vbox{\halign{##\cr
 \tiny\rightarrowfill\cr\noalign{\nointerlineskip\vskip1pt}
 $#1\mskip2mu$\cr}}}

%Format de fonctions:
\newcommand{\fct}[5]
	{
	  \begin{array}{ccccc}
		#1 & : & #2 & \to & #3 \\
	    && #4 & \mapsto & #5 \\
	  \end{array}
	}
\newcommand{\dfct}[2] {#1 \mapsto #2}

\renewcommand{\abs}[1]{|#1|}
\newcommand{\nm}[2]{ ||#1||_{#2} }
\begin{document}
\Large\underline{\textbf{Traduction: Machine à Pile}}\normalsize\\
$\llbracket E \rrbracket$: code pile évaluant E,le résultat va dans la pile\\
    \underline{Opération arithmétique avec Constantes}
    $\llbracket x +1 \rrbracket = $\\
    $\texttt{push } \delta(x)$\\
    $\texttt{mpush } 1$\\
    $\texttt{add}$\\
\rule{9cm}{0.1pt}\\
    \underline{Constantes}
    $\llbracket \texttt{cst} \rrbracket = $\\
    $\texttt{mpush cst}$\\
\rule{9cm}{0.1pt}\\
    \underline{Opération arithmétique}
    $\llbracket E1+E2 \rrbracket = $\\
    $\llbracket E1 \rrbracket$\\
    $\llbracket E2 \rrbracket$\\
    $\texttt{add}$\\
\rule{9cm}{0.1pt}\\
    \underline{Supérieur/Inférieur}
    $\llbracket E1 > (resp <) E2 \rrbracket = $\\
    $\llbracket E1 \rrbracket$\\
    $\llbracket E2 \rrbracket$\\
    $\texttt{testg}\hphantom{aa}(resp\hphantom{aa} \texttt{testl})$\\
\rule{9cm}{0.1pt}\\
    \underline{Affectation}
    $\llbracket x=E \rrbracket = $\\
    $\llbracket E \rrbracket$\\
    $\texttt{pop } \delta(x)$\\
\rule{9cm}{0.1pt}\\
    \underline{Contrôle}
    $\llbracket \texttt{if(C) S1 else S2} \rrbracket = $\\
    \hphantom{aaaaaa}$\llbracket C \rrbracket$\\
    \hphantom{aaaaaa}$\texttt{jz else}$\\
    \hphantom{aaaaaa}$\llbracket S1 \rrbracket$\\
    \hphantom{aaaaaa}$\texttt{jmp endif}$\\
    $\texttt{else:}$\\
    \hphantom{aaaaaa}$\llbracket S2 \rrbracket$\\
    $\texttt{endif:}$\\
\rule{9cm}{0.1pt}\\
    \underline{Programme}
    $\llbracket \texttt{init id1,\dots,idn; S1 ... Sp} \rrbracket = $\\
    $\texttt{alloc n}$\\
    $\llbracket S1 \rrbracket \dots \llbracket Sp \rrbracket$\\
    $\texttt{free}$\\
    $\texttt{stop}$\\
\rule{9cm}{0.1pt}\\
    \underline{Appel de fonction}
    $\llbracket \texttt{id(E1,\dots,En)} \rrbracket = $\\
    $\llbracket E1 \rrbracket \dots \llbracket En \rrbracket$\\
    $\texttt{call id}$\\
\rule{9cm}{0.1pt}\\
    \underline{Fonction}
    \small $\llbracket \texttt{int id(int id1\dots idp)}\lbrace\\
    \hphantom{aaa} \texttt{int id'1\dots id'q;S1\dots Sr} \rbrace \rrbracket =$\normalsize\\
    $\texttt{id:}$\\
    \hphantom{aaaaaa}$\texttt{alloc q}$\\
    \hphantom{aaaaaa}$\llbracket S1 \rrbracket \dots \llbracket Sr \rrbracket$\\
    \hphantom{aaaaaa}$\texttt{free}$\\
    \hphantom{aaaaaa}$\texttt{ret p}$\\
\rule{9cm}{0.1pt}\\
    \underline{Return}
    $\llbracket \texttt{return E} \rrbracket =$\\
    $\llbracket E \rrbracket$\\
    $\texttt{iret nb\_param}$\\
\rule{9cm}{0.1pt}\\
    \underline{Programme avec fonctions}
    $\llbracket \texttt{F1\dots Fp} \rrbracket = $\\
    $\texttt{call main}$\\
    $\texttt{mov rax,0}$\\
    $\texttt{ret 0}$\\
    $\llbracket F1 \rrbracket \dots \llbracket Fp \rrbracket$\\
\Large\underline{\textbf{Traduction: Machine à Registres}}\normalsize\\
\texttt{[t1+imm] = t2}: Ecrit t2 à l'adresse \texttt{t1+imm}\\
\texttt{t1 = [t2+imm]}: le contenu de l'adresse \texttt{t2+imm} dans t1\\
\rule{9cm}{0.1pt}
    \underline{Opération arithmétique}\\
    \texttt{t1 = imm} place \verb|imm| dans t1\\
    \texttt{t1 = t2} copie \verb|t2| dans \verb|t1|\\
\rule{9cm}{0.1pt}\\
    \underline{Tests}\\
    \texttt{t1 = t2 rel t3}; \\
    \verb|t1 = 1| si \texttt{t2 rel t3} 0 sinon\\
    \texttt{rel} peut être == / != / < / >\\
\rule{9cm}{0.1pt}
    \underline{Contrôle}\\
    \texttt{jump @} saute à \verb|@|\\
    \texttt{cjump t1 rel t2 --> @} saute à \verb|@| si t2 rel t1\\
\rule{9cm}{0.1pt}\\
$\llbracket E \rrbracket_t$ évalue E et le place dans le temporaire \verb|t|\\
$\rho$ donne un temporarire à partir d'une variable\\
    \underline{Expressions}\\
    $\llbracket \texttt{imm} \rrbracket_t = $\\
    \hphantom{aa}$\texttt{t = imm}$\\
    \rule{3cm}{0.1pt}\\
    $\llbracket \texttt{id} \rrbracket_t = $\\
    \hphantom{aa}$\texttt{t =} \rho(id)$\\
    \rule{3cm}{0.1pt}\\
    $\llbracket \texttt{E1 + E2} \rrbracket_t = $\\
    \hphantom{aa}$\llbracket \texttt{E1} \rrbracket_{t'}$\\
    \hphantom{aa}$\llbracket \texttt{E2} \rrbracket_{t''}$ \\
    \hphantom{aa}$\texttt{t = t' + t''}$\\
\rule{9cm}{0.1pt}\\
    \underline{Contrôle}\\
    $\llbracket \texttt{id = E} \rrbracket= $\\
    \hphantom{aa}$\llbracket \texttt{E} \rrbracket_{t} $\\
    \hphantom{aa}$\rho(id) \texttt{= t} $\\
    \rule{3cm}{0.1pt}\\
    $\llbracket \texttt{if(C) S1 else S2} \rrbracket =$\\
    \hphantom{aaaaaa}$\llbracket \texttt{C} \rrbracket_{t} $\\
    \hphantom{aaaaaa}$\texttt{cjump t == 0 --> else}$\\
    \hphantom{aaaaaa}$\llbracket \texttt{S1} \rrbracket$\\
    \hphantom{aaaaaa}$\texttt{jump endif}$\\
    $\texttt{else}:$\\
    \hphantom{aaaaaa}$\llbracket \texttt{S2} \rrbracket$\\
    $\texttt{endif}:$\\
\rule{9cm}{0.1pt}\\
    \underline{Construction des tableaux}\\
    $\llbracket \texttt{T[N1]\dots[Np]} \rrbracket_t= $\\
    \hphantom{aaaaaa}$\texttt{t = \$sp -1} $\\
    \hphantom{aaaaaa}$\texttt{\$sp = \$sp - N1}$\\
    \hphantom{aaaaaa}$\texttt{tc = t} $\\
    \hphantom{aaaaaa}$\texttt{tmin = \$sp}$\\
    $\texttt{loop:}$\\
    \hphantom{aaaaaa}$\texttt{cjump tc < tmin --> end}$\\
    \hphantom{aaaaaa}$\hphantom{aa}\llbracket \texttt{T[N2]\dots[Np]} \rrbracket_{t'}$\\
    \hphantom{aaaaaa}$\hphantom{aa}\texttt{[t-i+1] = t'}$\\
    \hphantom{aaaaaa}$\texttt{tc = tc -1}$\\
    \hphantom{aaaaaa}$\texttt{jump loop}$\\
    $\texttt{end:}$\\
\rule{9cm}{0.1pt}\\
\hphantom{a}\\
    \underline{Construction des structures}
    $\llbracket \texttt{struct}\lbrace \texttt{T1 id1;\dots; Tn idn;}\rbrace \rrbracket_t= $\\
    $\texttt{t = \$sp-1}$\\
    $\texttt{\$sp = \$sp-n}$\\
    $\llbracket \texttt{T1} \rrbracket_{t1}$\\
    $\texttt{[t] = t1} $\\
    $\vdots$\\
    $\llbracket \texttt{Tn} \rrbracket_{tn}$\\
    $\texttt{[t-n+1] = tn}$\\
\rule{9cm}{0.1pt}\\
    \underline{Conditions}
    $\llbracket \texttt{E1>E2} \rrbracket_t = $\\
    $\llbracket \texttt{E1} \rrbracket_{t'} = $\\
    $\llbracket \texttt{E2} \rrbracket_{t''} = $\\
    $\texttt{t = t' > t''}$\\
\rule{9cm}{0.1pt}\\
    \underline{Construction des scalaires}
    $\llbracket \texttt{T} \rrbracket_t= $\\
    $\texttt{t = 0} $\\
    $\llbracket \texttt{*T} \rrbracket_{t}= $\\
    $\texttt{t = 0}$ \\
\rule{9cm}{0.1pt}\\
    \underline{Return}
    $\llbracket \texttt{return E;} \rrbracket= $\\
    $\llbracket \texttt{E} \rrbracket_{t} $\\
    $\texttt{t' = t} $\\
    $\texttt{free}$\\
    $\texttt{ret nb\_parameters}$\\
\rule{9cm}{0.1pt}\\
    \underline{Affectations structurées}
    $\llbracket \texttt{E1 = E2} \rrbracket= $\\
    $\llbracket \texttt{E1} \rrbracket_{ta,tv}= $\\
    $\llbracket \texttt{E2} \rrbracket_{t}$\\
    $\texttt{[ta] = t}$\\    
\rule{9cm}{0.1pt}\\
    \underline{Appel de Fonction}
    $\llbracket \texttt{id(E1,\dots,En)} \rrbracket_t= $\\
    $\llbracket \texttt{E1} \rrbracket_{t1} $\\
    $\texttt{push t1} $\\
    $\vdots$\\
    $\llbracket \texttt{En} \rrbracket_{tn}$\\
    $\texttt{push tn} $\\
    $\texttt{call id}$\\
    $\texttt{t = t' (=rax)}$\\
\rule{9cm}{0.1pt}\\
    \underline{Fonctions}
    $\llbracket \texttt{T id(T1 id1,\dots,Tn idp)}\lbrace$\\
    \hphantom{aa} $\texttt{ T'1 id'1,\dots,T'q id'q;S1 \dots Sr}\rbrace \rrbracket= $\\
    $id:$\\
    \hphantom{aaaaaa}$\texttt{alloc ?} $\\
    \hphantom{aaaaaa}$\llbracket \texttt{T'1} \rrbracket_{t'1} \dots\llbracket \texttt{T'q} \rrbracket_{t'q} $\\
    \hphantom{aaaaaa}$\texttt{t1 = [\$bp+2+p-1]}$\\
    \hphantom{aaaaaa}$\vdots$\\
    \hphantom{aaaaaa}$\texttt{tp = [\$bp+2]}$\\
    \hphantom{aaaaaa}$\rho: \texttt{id1 --> t1 \dots idp --> tp}$\\
    \hphantom{aaaaaa}$\hphantom{aaa}\texttt{id'1 --> t'1 \dots id'q --> t'q}$\\
    \hphantom{aaaaaa}$\llbracket \texttt{S1} \rrbracket \dots\llbracket \texttt{Sr} \rrbracket$\\
    \hphantom{aaaaaa}$\texttt{free}$\\
    \hphantom{aaaaaa}$\texttt{ret p}$\\
\rule{9cm}{0.1pt}\\
    \underline{Accès tableau}
    $\llbracket \texttt{E1[E2]} \rrbracket_{ta,tv}= $\\
    $\llbracket \texttt{E1} \rrbracket_{tbase}= $\\
    $\llbracket \texttt{E2} \rrbracket_{toffset}=$\\
    $\texttt{ta = tbase - toffset}$\\
    $\texttt{tv = [ta]}$\\
\rule{9cm}{0.1pt}\\
    \underline{Accès struct}
    $\llbracket \texttt{E.id} \rrbracket_{ta,tv}= $\\
    $\llbracket \texttt{E} \rrbracket_{tbase}= $\\
    $\texttt{toffset = rang(id)}$\\
    $\texttt{ta = tbase - toffset}$\\
    $\texttt{tv = [ta]}$\\ 
\rule{9cm}{0.1pt}\\
    \underline{Accès pointeur}
    $\llbracket \texttt{*E} \rrbracket_{ta,tv}= $\\
    $\llbracket \texttt{E} \rrbracket_{ta}= $\\
    $\texttt{tv = [ta]}$ 
\end{document}
