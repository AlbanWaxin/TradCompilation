\documentclass[12pt,twocolumn]{report}
\usepackage[utf8]{inputenc}
\usepackage[T1]{fontenc}
\usepackage[fleqn]{amsmath}
\usepackage{amsfonts,amssymb,stmaryrd}
\usepackage[english]{babel}
\usepackage{pdfpages}

%=============Affichage=======================
\usepackage{fullpage}
\usepackage{mathtools}
\usepackage{lmodern}
\usepackage{xcolor}
\usepackage{enumitem}
\usepackage{tikz,tkz-tab}
\usepackage[ruled,vlined]{algorithm2e}
\usepackage{booktabs}
\usepackage{setspace}
\usepackage{tikz}
\usetikzlibrary{arrows,automata}
\title{Règles de traduction Compilation}
\date{}

\setlength{\topmargin}{-1.5cm}
\setlength{\textheight}{25cm}
\setlength{\textwidth}{20.5cm}
\setlength{\oddsidemargin}{-1.5cm}
\setlength{\evensidemargin}{50cm}
\doublespacing  

\newcommand{\rd}[1]{\textcolor{red}{#1}}
\newcommand{\g}[1]{\textcolor{lime}{#1}}
\newcommand{\dg}[1]{\textcolor{dgreen}{#1}}
\newcommand{\blue}[1]{\textcolor{blue}{#1}}
\newcommand{\cy}[1]{\textcolor{cyan}{#1}}
\newcommand{\blz}{$\blacklozenge$}
\newcommand{\ns}{\\\indent\indent\vspace{0.25cm}}
\setcounter{secnumdepth}{5}% profondeur de la table des matières
\usepackage{titlesec}


\titleformat{\chapter}[frame]
{\Huge}
{\filright\rmfamily\bfseries\Huge\enspace\thechapter\enspace}
{18pt}
{\rmfamily\huge\bfseries\filcenter}
% rmfamily=roman, sffamily = sans serif ou ttfamily =type writer
\usepackage[many]{tcolorbox} % Creation de box collorable pour le texte non intégré

\newtcolorbox{trad}{colback=white,colframe=black,arc=0mm,sharp corners= northwest,arc=10pt}

%=============Mathématiques=================

%--------------Raccourcis:------------------

\renewcommand{\o}{\circ}
\newcommand{\abs}[1]{\left\lvert#1\right\rvert}
\DeclarePairedDelimiter{\ceil}{\lceil}{\rceil}
\renewcommand*{\overrightarrow}[1]{\vbox{\halign{##\cr
 \tiny\rightarrowfill\cr\noalign{\nointerlineskip\vskip1pt}
 $#1\mskip2mu$\cr}}}

%Format de fonctions:
\newcommand{\fct}[5]
	{
	  \begin{array}{ccccc}
		#1 & : & #2 & \to & #3 \\
	    && #4 & \mapsto & #5 \\
	  \end{array}
	}
\newcommand{\dfct}[2] {#1 \mapsto #2}

\renewcommand{\abs}[1]{|#1|}
\newcommand{\nm}[2]{ ||#1||_{#2} }
\begin{document}
\chapter*{Règles de traduction: Machine à Pile}

$\llbracket E \rrbracket$ est le code pile qui évalue E et place le résultat dans la pile

\begin{trad}
\paragraph*{ \underline{Opération arithmétique avec Constantes}}
\begin{align*}
    &\llbracket x +1 \rrbracket = \\
    &\texttt{push } \delta(x)\\
    &\texttt{mpush } 1\\
    &\texttt{add}
\end{align*}
\end{trad}
\begin{trad}
    \paragraph*{ \underline{Constantes}}
    \begin{align*}
        &\llbracket \texttt{cst} \rrbracket = \\
        &\texttt{mpush cst}
    \end{align*}
\end{trad}
\begin{trad}
    \paragraph*{ \underline{Opération arithmétique}}
    \begin{align*}
        &\llbracket E1+E2 \rrbracket = \\
        &\llbracket E1 \rrbracket\\
        &\llbracket E2 \rrbracket\\
        &\texttt{add}
    \end{align*}
\end{trad}
\begin{trad}
    \paragraph*{ \underline{Supérieur}}
    \begin{align*}
        &\llbracket E1>E2 \rrbracket = \\
        &\llbracket E1 \rrbracket\\
        &\llbracket E2 \rrbracket\\
        &\texttt{testg}
    \end{align*}
\end{trad}
\begin{trad}
    \paragraph*{ \underline{Inférieur}}
    \begin{align*}
        &\llbracket E1<E2 \rrbracket = \\
        &\llbracket E1 \rrbracket\\
        &\llbracket E2 \rrbracket\\
        &\texttt{testl}
    \end{align*}
\end{trad}
\begin{trad}
    \paragraph*{ \underline{Affectation}}
    \begin{align*}
        &\llbracket x=E \rrbracket = \\
        &\llbracket E \rrbracket\\
        &\texttt{pop} \delta(x)
    \end{align*}
\end{trad}
\begin{trad}
    \paragraph*{ \underline{Contrôle}}
    \begin{align*}
        &\llbracket \texttt{if(C) S1 else S2} \rrbracket = \\
        &\llbracket C \rrbracket\\
        &\texttt{jz else}\\
        &\llbracket S1 \rrbracket\\
        &\texttt{jmp endif}\\
        \texttt{else:}&\\
        &\llbracket S2 \rrbracket\\
        \texttt{endif:}
    \end{align*}
\end{trad}
\begin{trad}
    \paragraph*{ \underline{Programme}}
    \begin{align*}
        &\llbracket \texttt{init id1,\dots,idn; S1 ... Sp} \rrbracket = \\
        &\texttt{alloc n}\\
        &\llbracket S1 \rrbracket \dots \llbracket Sp \rrbracket\\
        &\texttt{free}\\
        &\texttt{stop}
    \end{align*}
\end{trad}
\begin{trad}
    \paragraph*{ \underline{Appel de fonction}}
    \begin{align*}
        &\llbracket \texttt{id(E1,\dots,En)} \rrbracket = \\
        &\llbracket E1 \rrbracket \dots \llbracket En \rrbracket\\
        &\texttt{call id}\\
    \end{align*}
\end{trad}
\begin{trad}
    \paragraph*{ \underline{Fonction}}
    \begin{align*}
        &\llbracket \texttt{int id(int id1,\dots,int idp)}\lbrace \\
        &\texttt{int id'1,\dots,id'q;S1\dots Sr} \rbrace \rrbracket =\\
        \texttt{id:}&\\
        &\texttt{alloc q}\\
        &\llbracket S1 \rrbracket \dots \llbracket Sr \rrbracket\\
        &\texttt{free}\\
        &\texttt{ret p}
    \end{align*}
\end{trad}
\begin{trad}
    \paragraph*{ \underline{Return}}
    \begin{align*}
        &\llbracket \texttt{return E} \rrbracket = \\
        &\llbracket E \rrbracket\\
        &\texttt{iret nb\_param}\\
    \end{align*}
\end{trad}
\begin{trad}
    \paragraph*{ \underline{Programme avec fonctions}}
    \begin{align*}
        &\llbracket \texttt{F1\dots Fp} \rrbracket = \\
        &\texttt{call main}\\
        &\texttt{mov rax,0}\\
        &\texttt{ret 0}\\
        &\llbracket F1 \rrbracket\\
        & \dots\\
        &\llbracket Fp \rrbracket\\
    \end{align*}
\end{trad}

\chapter*{Règles de traduction: Machine à Registres}
\texttt{[t1 + imm] = t2} Ecrit t2 à l'adresse \texttt{t1+imm}

\texttt{ t1 = [t2 + imm]} Ecrit dans t1 la donnée stockée dans l'adresse \texttt{t2+imm}

\begin{trad}
    \paragraph*{ \underline{Opération arithmétique}}\hphantom{a}\\
    \texttt{t1 = imm} place \verb|imm| dans t1\\
    \texttt{t1 = t2} copie \verb|t2| dans \verb|t1|\\
\end{trad}
\begin{trad}
    \paragraph*{ \underline{Tests}}\hphantom{a}\\
    \texttt{t1 = t2 rel t3}; \\
    \verb|t1 = 1| si \texttt{t2 rel t3} 0 sinon\\
    \texttt{rel} peut être == / != / < / >\\
\end{trad}
\begin{trad}
    \paragraph*{ \underline{Contrôle}}\hphantom{a}\\
    \texttt{jump @} saute à \verb|@|\\
    \texttt{cjump t1 rel t2 --> @} saute à \verb|@| si t rel t1\\
\end{trad}
$\llbracket E \rrbracket_t$ évalue E et le place dans le temporaire \verb|t|\\
$\rho$ donne un temporarire à partir d'une variable\\
\begin{trad}
    \paragraph*{ \underline{Expressions}}
    \begin{align*}
        &\llbracket \texttt{imm} \rrbracket_t = \\
        &\texttt{t = imm}\\
        &\llbracket \texttt{id} \rrbracket_t = \\
        &\texttt{t =} \rho(id)\\
        &\llbracket \texttt{E1 + E2} \rrbracket_t = \\
        &\llbracket \texttt{E1} \rrbracket_{t'}\\
        &\llbracket \texttt{E2} \rrbracket_{t} = \\
        &\texttt{t = t' + t''}
    \end{align*}
\end{trad}
\begin{trad}
    \paragraph*{Construction des structures}
    \begin{align*}
        &\llbracket \texttt{struct{T1 id1;\dots; Tn idn;}} \rrbracket_t= \\
        &\texttt{t = \$sp-1}\\
        &\texttt{\$sp = \$sp-n}\\
        &\llbracket \texttt{T1} \rrbracket_{t1} \\
        &\texttt{[t] = t1} \\
        &\vdots
        &\llbracket \texttt{Tn} \rrbracket_{tn}\\
        &\texttt{[t-n+1] = tn} 
    \end{align*}
\end{trad}
\begin{trad}
    \paragraph*{Contrôle}
    \begin{align*}
        &\llbracket \texttt{id = E} \rrbracket= \\
        &\llbracket \texttt{E} \rrbracket_{t} \\
        &\rho(id) \texttt{= t} \\
        &\llbracket \texttt{if(C) S1 else S2} \rrbracket =\\
        &\llbracket \texttt{C} \rrbracket_{t} \\
        &\texttt{cjump t == 0 --> else}\\
        &\llbracket \texttt{S1} \rrbracket\\
        \texttt{else}:&\\
        &\llbracket \texttt{S2} \rrbracket\\
        \texttt{endif}:&
    \end{align*}
\end{trad}
\begin{trad}
    \paragraph*{Construction des tableaux}
    \begin{align*}
        &\llbracket \texttt{T[N1]\dots[Np]} \rrbracket_t= \\
        &\texttt{t = \$sp -1} \\
        &\texttt{\$sp = \$sp - N1}\\
        &\texttt{tc = t} \\
        &\texttt{tmin = \$sp}\\
        &\texttt{loop:}\\
        &\texttt{cjump tc < tmin --> end}\\
        &\hphantom{aa}\llbracket \texttt{T[N2]\dots[Np]} \rrbracket_{t'}\\
        &\hphantom{aa}\texttt{[t-i+1] = t'}\\
        &\texttt{tc = tc -1}\\
        &jump loop\\
        \texttt{end:}&
    \end{align*}
\end{trad}
\begin{trad}
    \paragraph*{Conditions}
    \begin{align*}
        &\llbracket \texttt{E1>E2} \rrbracket_t = \\
        &\llbracket \texttt{E1} \rrbracket_{t'} = \\
        &\llbracket \texttt{E2} \rrbracket_{t''} = \\
        &\texttt{t = t' > t''}
    \end{align*}
\end{trad}
\begin{trad}
    \paragraph*{Construction des scalaires}
    \begin{align*}
        &\llbracket \texttt{T} \rrbracket_t= \\
        &\texttt{t = 0} \\
        &\llbracket \texttt{*T} \rrbracket_{t}= \\
        &\texttt{t = 0} 
    \end{align*}
\end{trad}
\begin{trad}
    \paragraph*{Return}
    \begin{align*}
        &\llbracket \texttt{return E;} \rrbracket= \\
        &\llbracket \texttt{E} \rrbracket_{t} \\
        &\texttt{t' = t} \\
        &\texttt{free}\\
        &\texttt{ret nb\_parameters} 
    \end{align*}
\end{trad}
\begin{trad}
    \paragraph*{Affectations structurées}
    \begin{align*}
        &\llbracket \texttt{E1 = E2} \rrbracket= \\
        &\llbracket \texttt{E1} \rrbracket_{ta,tv}= \\
        &\llbracket \texttt{E2} \rrbracket_{t}\\
        &\texttt{[ta] = t} 
    \end{align*}
\end{trad}
\begin{trad}
    \paragraph*{Appel de Fonction}
    \begin{align*}
        &\llbracket \texttt{id(E1,\dots,En)} \rrbracket_t= \\
        &\llbracket \texttt{E1} \rrbracket_{t1} \\
        &\texttt{push t1} \\
        &\vdots\\
        &\llbracket \texttt{En} \rrbracket_{tn}\\
        &\texttt{push tn} \\
        &\texttt{call id}\\
        &\texttt{t = t'}
    \end{align*}
\end{trad}
\begin{trad}
    \paragraph*{Fonctions}
    \begin{align*}
        &\llbracket \texttt{T id(T1 id1,\dots,Tn idp)}\lbrace\\
        &\hphantom{aa} \texttt{ T'1 id'1,\dots,T'q id'q;S1 \dots Sr}\rbrace \rrbracket= \\
        id:&\\
        &\texttt{alloc ?} \\
        &\llbracket \texttt{T'1} \rrbracket_{t'1} \dots\llbracket \texttt{T'q} \rrbracket_{t'q} \\
        &\texttt{t1 = [\$bp+2+p-1]}\\
        &\vdots\\
        &\texttt{tp = [\$bp+2+p-1]}\\
        &\rho: \texttt{id1 --> t1 \dots idp --> tp}\\
        &\hphantom{aaa}\texttt{id'1 --> t'1 \dots id'q --> t'q}\\
        &\llbracket \texttt{S1} \rrbracket \dots\llbracket \texttt{Sr} \rrbracket\\
        &\texttt{free}\\
        &\texttt{ret p}
    \end{align*}
\end{trad}
\begin{trad}
    \paragraph*{Accès tableau}
    \begin{align*}
        &\llbracket \texttt{E1[E2]} \rrbracket_{ta,tv}= \\
        &\llbracket \texttt{E1} \rrbracket_{tbase}= \\
        &\llbracket \texttt{E2} \rrbracket_{toffset}= \\
        &\texttt{ta = tbase - toffset}\\
        &\texttt{tv = [ta]} 
    \end{align*}
\end{trad}
\begin{trad}
    \paragraph*{Accès struct}
    \begin{align*}
        &\llbracket \texttt{E.id} \rrbracket_{ta,tv}= \\
        &\llbracket \texttt{E} \rrbracket_{tbase}= \\
        &\texttt{toffset = rang(id)}\\
        &\texttt{ta = tbase - toffset}\\
        &\texttt{tv = [ta]} 
    \end{align*}
\end{trad}
\begin{trad}
    \paragraph*{Accès pointeur}
    \begin{align*}
        &\llbracket \texttt{*E} \rrbracket_{ta,tv}= \\
        &\llbracket \texttt{E} \rrbracket_{ta}= \\
        &\texttt{tv = [ta]} 
    \end{align*}
\end{trad}

\end{document}
